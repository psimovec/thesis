\documentclass{fithesis3}
\usepackage{amsmath}
\begin{document}
	\chapter{Introduction}
	\chapter{Preliminaries}
		\section{Büchi Automaton}
		A nondeterministic Büchi automaton (BA) is a tuple 
		$A=(\Sigma, Q, q_0, \Delta, \Gamma)$,
		where
		\begin{itemize}
			\item $\Sigma$ is a finite alphabet
			\item $Q$ is finite set of states
			\item $q_0 \in Q$ is the initial state
			\item $\Delta \subseteq Q \times \Sigma \times Q$
			are transitions
			\item $\Gamma \subseteq \Delta$
			is the transition-based acceptance condition
		\end{itemize} 
		\paragraph{run}
		A run $r$ of $A$ on $w \in \Sigma^\omega$
		is an $\omega$-word $r_0, w_0, r_1, w_1,...$ in $(Q\times\Sigma)^\omega$
		such that $r_0 = q_0 \land\forall i>0, (r_{i-1}, w_{i-1}, r_i) \in \Delta$ $o\omega o$
		\paragraph{inf(r)}
		We write $inf(r) \subseteq \Delta$ for the set of transitions that appear infinitely often in the run $r$.
		\paragraph{accepting run} A run $r$ is accepting if $inf(r) \cap \Gamma \neq \emptyset$
		\paragraph{language}
		The language $L_A\subseteq\Sigma^\omega$ is recognized by $A$.\newline
		$\forall w \in L_A \exists r$ on $w$ such that r is accepting.
		\paragraph{$\omega$-regular language}
		A language is $\omega$-regular if it is accepted by BA.
		\paragraph{deterministic automaton}
		$A=(\Sigma, Q, q_0, \Delta, \Gamma)$ is deterministic if \newline$(q, \rho, q^`), (q, \rho, q^{``})\in\Delta\implies q^`=q^{``}$
		\paragraph{complete automaton}
		$A$ is complete if, $\forall w \in \Sigma, \forall q \in Q, \exists (q, w, q^`)\in\Delta$. A word in $\Sigma^\omega$ has exactly one run in a deterministic, complete automaton.
		
		\section{Markov Decision Processes}
		A Markov decision process (MDP) $M$
		is a tuple $(S, A, T, \Sigma, L)$, where
		\begin{itemize}
			\item $S$ is a finite set of states
			\item $A$ is a finite set of actions
			\item $T:S\times A \rightarrow D(S)$, where $D(S)$ is set of probability distributions over S, is  the probabilistic transition (partial) function
			\item $\Sigma$ is an alphabet
			\item $L:S\times A \times S \rightarrow \Sigma$ is the labeling function of the set of transitions.
			For a state $s \in S, A(s)$ denotes the set of actions available in s.
		\end{itemize}
		\paragraph{run}
		A run of $M$ is an $\omega$-word 
		$s_0,a_1,...\in A=S \times (A \times S)^\omega$
		such that $Pr(s_{i+1}|s_i, a_{i+1})>0$ for all $i >= 0$. A finite run is a finite such sequence.
		\paragraph{labeled run}
		We define labeled run as
		$L(r)=L(s_0,a_1,s_1), L(s_1, a_2, s_2),... \in \Sigma^\omega$.
		\paragraph{paths}
		We write $\Omega(M)(Paths(M))$
		for the set of runs (finite runs) of $M$ and
		$\Omega_s(M)(Paths_s(M))$ for the set of runs (finite runs)
		of $M$ starting from state $s$. When the $MDP$ is clear from the context we drop the argument $M$.
		\paragraph{strategy}
		A strategy in $M$ is a
		function $\mu:Paths \rightarrow D(A)$
		such that $supp(\mu(r))\subseteq A(last(r))$,
		where $supp(d)$ is the support of $d$ and
		$last(r)$ is the last state of $r$.
		Let $\Omega^M_\mu$ denote the subset of runs $\Omega^M$
		that correspond to strategy $\mu$ and initial state $s$.
		Let $\Pi_M$ be the set of all strategies.
		\paragraph{pure strategy}
		We say that a strategy $\mu$ is pure
		if $\mu(r)$ is a point distribution for all runs $r \in Paths$.
		\paragraph{behavior}
		The behavior of an MDP $M$
		under a strategy $\mu$ with starting state $s$ is defined
		on a probability space $(\Omega_s^\mu, F_s^\mu, Pr_s^\mu)$
		over the set of infinite runs of $\mu$ from $s$.
		
		\section{Good-for-MDP (GFM) Automata}
		
		\section{to be defined}
		$\omega$-word?, point distribution?, what is $F_s^\mu$ in 'pure strategy' paragraph?, TGBA, describe Semi-determistic as I am going to compare them with SBA 
		\subsection{xd}
		
		,
		,
		 
		
		GF MDP, model checking
		\section{Algorithms}
		BP + both slim 
	\chapter{Implementation}
		\section{Technologies}
		\section{Implementation inside Seminator}
	\chapter{Evaluation}
		\section{Alternative Algorithm}
		\section{Different Implementation - ePMC}
		\section{Semi-deterministic Automata}
	\chapter{Conclusion}
\end{document}