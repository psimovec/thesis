\documentclass[
]{beamer}

\usepackage[czech]{babel}
\usepackage[utf8]{inputenc}
\usepackage[T1]{fontenc}
\usepackage{fontspec}
\usepackage{csquotes}
\usepackage{biblatex}
\usepackage{tikz}
\usepackage{pgfplots}
\usepackage[nodayofweek,level]{datetime}
\newcommand{\hlineny}{\hline}
\addbibresource{exampl.bib}
\usepackage{booktabs}
\usetheme[
workplace=fi,
]{}

\title[Slim Automata]{Transformace Nedeterministických Büchiho Automatů na Slim Automaty}
\author[P. Šimovec]{Pavel Šimovec\texorpdfstring{\\}{, }}
\institute[FI MU]{Fakulta Informatiky, Masarykova Univerzita}
\date{\formatdate{30}{6}{2021}}
\subject{Presentation Subject}
\keywords{the, presentation, keywords}



\DeclareCiteCommand{\citetitleyear}
{\boolfalse{citetracker}%
	\boolfalse{pagetracker}}
{\ifciteindex
	{\indexfield{indextitle}}
	{}%
	\printfield[citetitle]{labeltitle}%
	\setunit{\addspace}%
	\printtext[parens]{%
		\usebibmacro{prenote}%
		\printfield{year}\printfield{extrayear}%
		\usebibmacro{postnote}}}
{\multicitedelim}
{}




\setbeamertemplate{navigation symbols}{%
	\usebeamerfont{footline}%
	\usebeamercolor[fg]{footline}%
	\hspace{1em}%
	\insertframenumber/\inserttotalframenumber
}
\setbeamercolor{footline}{fg=black}
\setbeamerfont{footline}{series=\bfseries}

\begin{document}
	
	\begin{frame}[plain]
		\maketitle
	\end{frame}
	
	
	\begin{frame}{Motivace}
		\begin{itemize}
			\pause
			\item Nedeterministické Büchiho automaty obecně nejsou \alert{dobré pro Markovovy rozhodovací procesy} (dále \alert{GFM}, good for Markov decision processes) $\rightarrow$ nejsou vhodné pro analýzu pravděpodobnostních systémů
			\pause
			\item Deterministické Büchiho automaty jsou obecně GFM, ale mohou mít větší velikost či složitější akceptační podmínku
			\pause
			\item Motivace je vyhnout se determinizaci a zároveň mít GFM automaty
			\pause
			\item Využití ve zpětnovazebném učení - nástroj \alert{Mungojerrie} vyžaduje GFM automaty
			\pause
			\item Motivace zrychlit zpětnovazebné učení
			\pause
			\item Slim automaty jsou nedeterministické a zároveň mají GFM vlastnost
		\end{itemize}
	\end{frame}

	\begin{frame}{Přínos v teorii}
		\begin{itemize}
			\pause
			\item Slim automaty byly popsány ve článku \citetitleyear{hlavni}
			\pause
			\item Rozšíření algoritmu z článku na \alert{generalizované} Büchiho automaty (GBA)
			\pause
			\begin{itemize}
			\item Bez tohoto rozšíření je třeba prvně provést degeneralizaci
			\end{itemize}
			\pause
			\item První formální popis \alert{weak slim automatů} (slim automaty z původního článku pojmenovány jako \alert{strong})
		\end{itemize}
	
	
	\end{frame}
	
	\begin{frame}{Implementace}
		\begin{itemize}
			\pause
			\item Implementace algoritmu z původního článku a rozšíření na GBA v nástroji Seminator
			\pause
			\item Weak/strong a via-tba/via-tgba 
			\pause
			\item Přidána možnost post-optimalizace pomocí knihovny SPOT	
			\pause
			\item Testována jazyková ekvivalence se zdrojovým automatem
			\pause
			\item V době implementace nebyly k dispozici jiné implementace (později nástroj ePMC)
			
		\end{itemize}
	\end{frame}

	\begin{frame}{Evaluace}
		Automaty porovnáváme 2 způsoby:
		\pause
		\begin{itemize}
			\item porovnání počtu stavů automatů
			\begin{itemize}
				\item mezi implementovanými možnostmi
				\item s jinými nástroji
			\end{itemize}
			\pause
			\item vliv na zpětnovazebné učení v nástroji Mungojerrie
		\end{itemize}
	\end{frame}



	\begin{frame}{Porovnání počtu stavů automatů}
		\begin{table}[ht]
			\centering
			\caption{Porovnání kumulativního počtu stavů slim automatů tvořených Seminatorem}
	\begin{tabular}{ |c||c|c|c| } 		
		\hline
		
		slim&weak&\alert{strong}&nejlepší \\
		\hline
		\alert{via tba}&9160& \alert{7725} & 7456\\
		\hline
		via tgba&10317& 8793& 8434\\ 
		\hline
		nejlepší&8958& 7578& \alert{7275} \\ 
		\hline
	\end{tabular}
\end{table}
\end{frame}

	\begin{frame}{Porovnání počtu stavů automatů}
		\begin{table}[ht]
			\centering
			
		
		
		\begin{tabular}{|c||c|}
			\hline
			nástroj  &   kumulativní počet stavů \\
			\hlineny
			Seminator slim          &        7133 \\
			
			\hline
			ePMC     &     10132 \\
			\hline
			ltl2ldba      &        4632 \\
			\hlineny
			ltl3tela (deterministické)&  	3975\\\hlineny
			
		\end{tabular}
\end{table}
	\end{frame}
	\begin{frame}{Porovnání počtu stavů automatů}
		\begin{figure}[ht]
			\centering
			
			\caption{Scatter plot porovnávajíci velikosti slim automatů. }
			\label{scatter:slim-epmc}
			\begin{tikzpicture}
				\pgfplotsset{
  compat=newest,
}
\begin{axis}[
  mark size=1.2pt,
  axis x line*=bottom,
  axis y line*=left,
  width=7cm,
  height=6.5cm,
  xlabel near ticks=true,
  ylabel near ticks=true,
  xmin=0,
  ymin=0,
  colorbar/width=.1cm,
  colorbar style={
    line width=.1pt,
  },
  colorbar shift/.style={
    xshift=.1cm,
  },
  xlabel={yes.slim},
  ylabel={yes.epmcbest},
%
]
\addplot[
  scatter=false,
  scatter src=explicit,
  only marks=true,
  mark options={
    fill opacity=0.3,
    draw opacity=0,
  },
  every mark/.append style={},
%
] coordinates
  {(17.0,26.0) [1]%
  (30.0,36.0) [1]%
  (9.0,10.0) [1]%
  (11.0,14.0) [1]%
  (13.0,23.0) [1]%
  (10.0,14.0) [1]%
  (14.0,24.0) [1]%
  (52.0,53.0) [1]%
  (5.0,6.0) [1]%
  (6.0,7.0) [1]%
  (8.0,15.0) [1]%
  (9.0,10.0) [1]%
  (17.0,21.0) [1]%
  (12.0,19.0) [1]%
  (4.0,3.0) [1]%
  (7.0,8.0) [1]%
  (3.0,4.0) [1]%
  (56.0,115.0) [1]%
  (27.0,37.0) [1]%
  (7.0,10.0) [1]%
  (14.0,19.0) [1]%
  (6.0,7.0) [1]%
  (4.0,5.0) [1]%
  (16.0,20.0) [1]%
  (15.0,17.0) [1]%
  (4.0,10.0) [1]%
  (5.0,6.0) [1]%
  (10.0,11.0) [1]%
  (20.0,26.0) [1]%
  (10.0,15.0) [1]%
  (8.0,11.0) [1]%
  (8.0,18.0) [1]%
  (10.0,12.0) [1]%
  (4.0,5.0) [1]%
  (11.0,12.0) [1]%
  (3.0,6.0) [1]%
  (10.0,12.0) [1]%
  (14.0,20.0) [1]%
  (5.0,6.0) [1]%
  (31.0,147.0) [1]%
  (2.0,3.0) [1]%
  (48.0,70.0) [1]%
  (16.0,19.0) [1]%
  (7.0,8.0) [1]%
  (6.0,17.0) [1]%
  (5.0,13.0) [1]%
  (15.0,26.0) [1]%
  (16.0,25.0) [1]%
  (26.0,36.0) [1]%
  (4.0,3.0) [1]%
  (84.0,nan) [1]%
  (8.0,12.0) [1]%
  (12.0,15.0) [1]%
  (15.0,26.0) [1]%
  (12.0,21.0) [1]%
  (15.0,26.0) [1]%
  (26.0,32.0) [1]%
  (6.0,9.0) [1]%
  (6.0,13.0) [1]%
  (8.0,9.0) [1]%
  (6.0,9.0) [1]%
  (7.0,8.0) [1]%
  (89.0,126.0) [1]%
  (5.0,6.0) [1]%
  (nan,316.0) [1]%
  (6.0,9.0) [1]%
  (9.0,7.0) [1]%
  (7.0,10.0) [1]%
  (6.0,9.0) [1]%
  (187.0,238.0) [1]%
  (4.0,8.0) [1]%
  (74.0,112.0) [1]%
  (14.0,15.0) [1]%
  (14.0,18.0) [1]%
  (9.0,10.0) [1]%
  (6.0,7.0) [1]%
  (13.0,14.0) [1]%
  (26.0,66.0) [1]%
  (6.0,7.0) [1]%
  (4.0,11.0) [1]%
  (71.0,73.0) [1]%
  (16.0,17.0) [1]%
  (24.0,25.0) [1]%
  (4.0,5.0) [1]%
  (10.0,19.0) [1]%
  (6.0,4.0) [1]%
  (3.0,10.0) [1]%
  (9.0,13.0) [1]%
  (9.0,14.0) [1]%
  (6.0,10.0) [1]%
  (11.0,13.0) [1]%
  (13.0,10.0) [1]%
  (9.0,13.0) [1]%
  (16.0,40.0) [1]%
  (5.0,6.0) [1]%
  (4.0,5.0) [1]%
  (5.0,4.0) [1]%
  (11.0,12.0) [1]%
  (8.0,12.0) [1]%
  (37.0,44.0) [1]%
  (8.0,10.0) [1]%
  (13.0,20.0) [1]%
  (142.0,nan) [1]%
  (8.0,11.0) [1]%
  (51.0,59.0) [1]%
  (17.0,16.0) [1]%
  (12.0,15.0) [1]%
  (23.0,22.0) [1]%
  (11.0,16.0) [1]%
  (6.0,7.0) [1]%
  (22.0,21.0) [1]%
  (7.0,8.0) [1]%
  (9.0,18.0) [1]%
  (14.0,15.0) [1]%
  (12.0,16.0) [1]%
  (15.0,14.0) [1]%
  (16.0,25.0) [1]%
  (8.0,12.0) [1]%
  (21.0,22.0) [1]%
  (7.0,16.0) [1]%
  (9.0,12.0) [1]%
  (20.0,27.0) [1]%
  (10.0,11.0) [1]%
  (7.0,19.0) [1]%
  (8.0,12.0) [1]%
  (13.0,16.0) [1]%
  (7.0,8.0) [1]%
  (4.0,3.0) [1]%
  (38.0,54.0) [1]%
  (7.0,8.0) [1]%
  (8.0,17.0) [1]%
  (19.0,27.0) [1]%
  (4.0,10.0) [1]%
  (6.0,8.0) [1]%
  (26.0,18.0) [1]%
  (8.0,9.0) [1]%
  (15.0,22.0) [1]%
  (7.0,18.0) [1]%
  (11.0,16.0) [1]%
  (6.0,9.0) [1]%
  (6.0,7.0) [1]%
  (6.0,7.0) [1]%
  (14.0,17.0) [1]%
  (6.0,22.0) [1]%
  (17.0,18.0) [1]%
  (4.0,3.0) [1]%
  (12.0,19.0) [1]%
  (8.0,10.0) [1]%
  (7.0,8.0) [1]%
  (5.0,7.0) [1]%
  (8.0,11.0) [1]%
  (24.0,143.0) [1]%
  (12.0,17.0) [1]%
  (8.0,14.0) [1]%
  (10.0,11.0) [1]%
  (16.0,20.0) [1]%
  (16.0,27.0) [1]%
  (8.0,10.0) [1]%
  (30.0,38.0) [1]%
  (8.0,9.0) [1]%
  (18.0,26.0) [1]%
  (7.0,13.0) [1]%
  (5.0,27.0) [1]%
  (5.0,19.0) [1]%
  (5.0,9.0) [1]%
  (33.0,47.0) [1]%
  (13.0,15.0) [1]%
  (27.0,55.0) [1]%
  (10.0,11.0) [1]%
  (14.0,32.0) [1]%
  (10.0,9.0) [1]%
  (19.0,54.0) [1]%
  (45.0,58.0) [1]%
  (8.0,10.0) [1]%
  (38.0,51.0) [1]%
  (7.0,9.0) [1]%
  (7.0,8.0) [1]%
  (20.0,19.0) [1]%
  (12.0,10.0) [1]%
  (9.0,10.0) [1]%
  (6.0,5.0) [1]%
  (12.0,50.0) [1]%
  (17.0,32.0) [1]%
  (7.0,8.0) [1]%
  (18.0,26.0) [1]%
  (20.0,30.0) [1]%
  (68.0,71.0) [1]%
  (15.0,36.0) [1]%
  (31.0,37.0) [1]%
  (6.0,7.0) [1]%
  (9.0,10.0) [1]%
  (7.0,20.0) [1]%
  (7.0,8.0) [1]%
  (24.0,32.0) [1]%
  (13.0,10.0) [1]%
  (9.0,11.0) [1]%
  (17.0,27.0) [1]%
  (16.0,49.0) [1]%
  (42.0,36.0) [1]%
  (7.0,8.0) [1]%
  (30.0,35.0) [1]%
  (11.0,12.0) [1]%
  (6.0,7.0) [1]%
  (37.0,50.0) [1]%
  (12.0,14.0) [1]%
  (30.0,33.0) [1]%
  (43.0,66.0) [1]%
  (11.0,12.0) [1]%
  (nan,nan) [1]%
  (16.0,12.0) [1]%
  (5.0,9.0) [1]%
  (24.0,36.0) [1]%
  (14.0,13.0) [1]%
  (13.0,19.0) [1]%
  (101.0,144.0) [1]%
  (12.0,17.0) [1]%
  (26.0,43.0) [1]%
  (10.0,11.0) [1]%
  (21.0,32.0) [1]%
  (10.0,16.0) [1]%
  (7.0,18.0) [1]%
  (33.0,45.0) [1]%
  (13.0,16.0) [1]%
  (11.0,10.0) [1]%
  (34.0,51.0) [1]%
  (22.0,43.0) [1]%
  (63.0,78.0) [1]%
  (9.0,29.0) [1]%
  (5.0,9.0) [1]%
  (10.0,13.0) [1]%
  (9.0,10.0) [1]%
  (11.0,17.0) [1]%
  (16.0,24.0) [1]%
  (9.0,13.0) [1]%
  (11.0,16.0) [1]%
  (5.0,6.0) [1]%
  (16.0,20.0) [1]%
  (23.0,28.0) [1]%
  (52.0,74.0) [1]%
  (7.0,16.0) [1]%
  (11.0,20.0) [1]%
  (13.0,11.0) [1]%
  (14.0,20.0) [1]%
  (32.0,46.0) [1]%
  (13.0,21.0) [1]%
  (22.0,23.0) [1]%
  (5.0,6.0) [1]%
  (11.0,21.0) [1]%
  (4.0,5.0) [1]%
  (7.0,8.0) [1]%
  (6.0,8.0) [1]%
  (13.0,15.0) [1]%
  (29.0,43.0) [1]%
  (19.0,25.0) [1]%
  (5.0,6.0) [1]%
  (8.0,9.0) [1]%
  (12.0,57.0) [1]%
  (9.0,10.0) [1]%
  (17.0,28.0) [1]%
  (12.0,13.0) [1]%
  (4.0,5.0) [1]%
  (9.0,15.0) [1]%
  (4.0,5.0) [1]%
  (16.0,17.0) [1]%
  (10.0,12.0) [1]%
  (11.0,27.0) [1]%
  (46.0,77.0) [1]%
  (9.0,19.0) [1]%
  (8.0,10.0) [1]%
  (7.0,8.0) [1]%
  (8.0,16.0) [1]%
  (4.0,10.0) [1]%
  (18.0,37.0) [1]%
  (49.0,54.0) [1]%
  (8.0,9.0) [1]%
  (28.0,46.0) [1]%
  (8.0,9.0) [1]%
  (13.0,14.0) [1]%
  (7.0,8.0) [1]%
  (16.0,22.0) [1]%
  (3.0,11.0) [1]%
  (6.0,5.0) [1]%
  (10.0,11.0) [1]%
  (15.0,11.0) [1]%
  (15.0,20.0) [1]%
  (25.0,28.0) [1]%
  (6.0,8.0) [1]%
  (11.0,14.0) [1]%
  (7.0,9.0) [1]%
  (5.0,7.0) [1]%
  (26.0,28.0) [1]%
  (18.0,19.0) [1]%
  (24.0,52.0) [1]%
  (7.0,8.0) [1]%
  (18.0,30.0) [1]%
  (20.0,27.0) [1]%
  (6.0,9.0) [1]%
  (16.0,23.0) [1]%
  (17.0,24.0) [1]%
  (11.0,17.0) [1]%
  (7.0,8.0) [1]%
  (4.0,5.0) [1]%
  (9.0,10.0) [1]%
  (14.0,18.0) [1]%
  (23.0,34.0) [1]%
  (13.0,20.0) [1]%
  (192.0,373.0) [1]%
  (13.0,25.0) [1]%
  (16.0,21.0) [1]%
  (7.0,8.0) [1]%
  (126.0,265.0) [1]%
  (24.0,23.0) [1]%
  (8.0,10.0) [1]%
  (18.0,22.0) [1]%
  (23.0,29.0) [1]%
  (16.0,26.0) [1]%
  (10.0,14.0) [1]%
  (9.0,15.0) [1]%
  (45.0,62.0) [1]%
  (11.0,8.0) [1]%
  (9.0,15.0) [1]%
  (8.0,9.0) [1]%
  (8.0,9.0) [1]%
  (42.0,73.0) [1]%
  (7.0,8.0) [1]%
  (5.0,11.0) [1]%
  (6.0,8.0) [1]%
  (17.0,20.0) [1]%
  (4.0,3.0) [1]%
  (8.0,10.0) [1]%
  (12.0,13.0) [1]%
  (13.0,15.0) [1]%
  (67.0,212.0) [1]%
  (4.0,10.0) [1]%
  (17.0,20.0) [1]%
  (22.0,29.0) [1]%
  (144.0,221.0) [1]%
  (3.0,4.0) [1]%
  (8.0,35.0) [1]%
  (8.0,12.0) [1]%
  (3.0,6.0) [1]%
  (83.0,180.0) [1]%
  (65.0,96.0) [1]%
  (26.0,31.0) [1]%
  (14.0,13.0) [1]%
  (23.0,30.0) [1]%
  (30.0,149.0) [1]%
  (4.0,9.0) [1]%
  (7.0,5.0) [1]%
  (10.0,12.0) [1]%
  (31.0,43.0) [1]%
  (4.0,6.0) [1]%
  (8.0,9.0) [1]%
  (18.0,25.0) [1]%
  (39.0,70.0) [1]%
  (7.0,10.0) [1]%
  (23.0,31.0) [1]%
  (13.0,23.0) [1]%
  (59.0,62.0) [1]%
  (24.0,34.0) [1]%
  (7.0,5.0) [1]%
  (6.0,8.0) [1]%
  (7.0,9.0) [1]%
  (51.0,47.0) [1]%
  (9.0,13.0) [1]%
  (5.0,6.0) [1]%
  (12.0,13.0) [1]%
  (18.0,30.0) [1]%
  (5.0,9.0) [1]%
  (8.0,12.0) [1]%
  (25.0,18.0) [1]%
  (6.0,19.0) [1]%
  (24.0,34.0) [1]%
  (7.0,8.0) [1]%
  (28.0,47.0) [1]%
  (11.0,12.0) [1]%
  (5.0,4.0) [1]%
  (23.0,25.0) [1]%
  (5.0,6.0) [1]%
  (12.0,20.0) [1]%
};%
\addplot[gray,domain=0:192.0]{x};%
%
\end{axis}

			\end{tikzpicture}
		\end{figure}
	\end{frame}
	
	
\begin{frame}{Porovnání vlivu na zpětnovazebné učení}
	\begin{itemize}
		\pause
		\item \alert{Mungojerrie} je nástroj pro zpětnovazebné učení vydaný letos (rok 2021)\pause
		\item zpětnovazebné učení v Mungojerrie má 2 fáze\pause
		\begin{itemize}
			\item fázi učení s daným počtem epizod
			\item fázi model checkingu
		\end{itemize}\pause
		\item \alert{cíl je definován GFM automatem}, který nástroji dodáme\pause
		\item měříme po kolika epizodách máme pravděpodobnost dosažení cíle rovnou jedné\pause
		\item meříme desetkrát pro seedy 0-9\pause
		\item v případě neúspěchu dosažení pravděpodobnosti 1 u jednoho seedu považujeme celý experiment pro automat za neúspěšný.
	\end{itemize}
		
\end{frame}
\begin{frame}{Porovnání vlivu na zpětnovazebné učení}
	\begin{tikzpicture}
\pgfplotsset{
	compat=newest,
	scaled y ticks=false
}
\begin{axis}[
	very thick=true,
	no markers=true,
	axis x line*=bottom,
	axis y line*=left,
	width=12cm,
	height=15cm,
	cycle list={%
		{green, solid},
		{blue, densely dashed},
		{red, dashdotdotted},
		{black, densely dotted},
		{brown, loosely dashdotted}
	},
	xlabel near ticks=true,
	ylabel near ticks=true,
	xmin=0,
	ymin=-1000,
	legend pos=north west,
	every axis legend/.append style={
		cells={
			anchor=west,
		},
		draw=none,
	},
	xmax=343,
	ymax=23040.0,
	%
	]
	\addplot coordinates {(0,1.0) (1,1.0) (2,1.0) (3,1.0) (4,1.0) (5,1.0) (6,1.0) (7,1.0) (8,1.0) (9,1.0) (10,1.0) (11,1.0) (12,1.0) (13,1.0) (14,1.0) (15,1.0) (16,1.0) (17,1.0) (18,1.0) (19,1.0) (20,1.0) (21,1.0) (22,1.0) (23,1.0) (24,1.0) (25,1.0) (26,1.0) (27,1.0) (28,1.0) (29,1.0) (30,1.0) (31,1.0) (32,1.0) (33,1.0) (34,1.0) (35,1.0) (36,1.0) (37,1.0) (38,1.0) (39,1.0) (40,1.0) (41,1.0) (42,1.0) (43,1.0) (44,1.0) (45,1.0) (46,1.0) (47,1.0) (48,1.0) (49,1.0) (50,1.0) (51,1.0) (52,1.0) (53,1.0) (54,1.0) (55,1.0) (56,1.0) (57,1.0) (58,1.0) (59,1.0) (60,1.0) (61,1.0) (62,1.0) (63,1.0) (64,1.0) (65,1.0) (66,1.0) (67,1.0) (68,1.0) (69,1.0) (70,1.0) (71,1.0) (72,1.0) (73,1.0) (74,1.0) (75,1.0) (76,1.0) (77,1.0) (78,1.0) (79,1.0) (80,1.0) (81,1.0) (82,1.0) (83,1.0) (84,1.0) (85,1.0) (86,1.0) (87,1.0) (88,1.0) (89,1.0) (90,3.0) (91,3.0) (92,5.0) (93,5.0) (94,7.0) (95,13.0) (96,15.0) (97,15.0) (98,15.0) (99,15.0) (100,15.0) (101,15.0) (102,15.0) (103,15.0) (104,31.0) (105,43.0) (106,43.0) (107,45.0) (108,45.0) (109,62.0) (110,68.0) (111,68.0) (112,68.0) (113,68.0) (114,68.0) (115,90.0) (116,141.0) (117,141.0) (118,149.0) (119,149.0) (120,149.0) (121,149.0) (122,149.0) (123,149.0) (124,149.0) (125,180.0) (126,180.0) (127,180.0) (128,180.0) (129,269.0) (130,269.0) (131,269.0) (132,269.0) (133,269.0) (134,284.0) (135,344.0) (136,344.0) (137,344.0) (138,344.0) (139,352.0) (140,360.0) (141,360.0) (142,360.0) (143,360.0) (144,360.0) (145,360.0) (146,360.0) (147,360.0) (148,486.0) (149,524.0) (150,524.0) (151,524.0) (152,524.0) (153,524.0) (154,561.0) (155,636.0) (156,636.0) (157,636.0) (158,636.0) (159,636.0) (160,636.0) (161,644.0) (162,651.0) (163,659.0) (164,681.0) (165,720.0) (166,720.0) (167,720.0) (168,720.0) (169,720.0) (170,720.0) (171,720.0) (172,720.0) (173,720.0) (174,720.0) (175,720.0) (176,720.0) (177,720.0) (178,720.0) (179,720.0) (180,720.0) (181,966.0) (182,966.0) (183,966.0) (184,966.0) (185,966.0) (186,989.0) (187,1034.0) (188,1034.0) (189,1034.0) (190,1034.0) (191,1034.0) (192,1041.0) (193,1056.0) (194,1056.0) (195,1056.0) (196,1056.0) (197,1056.0) (198,1079.0) (199,1079.0) (200,1079.0) (201,1079.0) (202,1079.0) (203,1086.0) (204,1086.0) (205,1124.0) (206,1184.0) (207,1221.0) (208,1259.0) (209,1259.0) (210,1259.0) (211,1259.0) (212,1274.0) (213,1341.0) (214,1364.0) (215,1364.0) (216,1364.0) (217,1364.0) (218,1440.0) (219,1440.0) (220,1440.0) (221,1440.0) (222,1440.0) (223,1440.0) (224,1440.0) (225,1440.0) (226,1440.0) (227,1440.0) (228,1440.0) (229,1440.0) (230,1440.0) (231,1440.0) (232,1440.0) (233,1440.0) (234,1440.0) (235,1440.0) (236,1440.0) (237,1440.0) (238,1440.0) (239,1440.0) (240,1440.0) (241,1440.0) (242,1440.0) (243,1440.0) (244,1986.0) (245,1986.0) (246,1986.0) (247,1986.0) (248,2024.0) (249,2024.0) (250,2031.0) (251,2039.0) (252,2076.0) (253,2136.0) (254,2369.0) (255,2406.0) (256,2406.0) (257,2406.0) (258,2826.0) (259,2826.0) (260,2834.0) (261,2834.0) (262,2880.0) (263,2880.0) (264,2880.0) (265,2880.0) (266,2880.0) (267,2880.0) (268,2880.0) (269,2880.0) (270,2880.0) (271,2880.0) (272,2880.0) (273,2880.0) (274,2880.0) (275,3996.0) (276,3996.0) (277,3996.0) (278,4071.0) (279,4071.0) (280,4071.0) (281,4319.0) (282,4319.0) (283,4319.0) (284,4349.0) (285,4409.0) (286,4409.0) (287,4454.0) (288,4949.0) (289,4949.0) (290,4949.0) (291,5399.0) (292,5760.0) (293,5760.0) (294,5760.0) (295,5760.0) (296,5760.0) (297,5760.0) (298,5760.0) (299,5760.0) (300,5760.0) (301,5760.0) (302,5760.0) (303,8691.0) (304,8691.0) (305,8871.0) (306,15854.0) (307,16056.0) (308,16386.0) (309,16401.0) (310,16476.0) (311,16499.0) (312,16506.0) (313,16581.0) (314,16604.0) (315,16634.0) (316,16739.0) (317,16769.0) (318,21899.0) (319,22146.0) (320,22244.0) (321,22259.0) (322,22439.0) (323,23040.0) (324,23040.0) (325,23040.0) (326,23040.0) (327,23040.0) (328,nan) (329,nan) (330,nan) (331,nan) (332,nan) (333,nan) (334,nan) (335,nan) (336,nan) (337,nan) (338,nan) (339,nan) (340,nan) (341,nan) (342,nan)};%
	\addlegendentry{slims/}%
	\addplot coordinates {(0,1.0) (1,1.0) (2,1.0) (3,1.0) (4,1.0) (5,1.0) (6,1.0) (7,1.0) (8,1.0) (9,1.0) (10,1.0) (11,1.0) (12,1.0) (13,1.0) (14,1.0) (15,1.0) (16,1.0) (17,1.0) (18,1.0) (19,1.0) (20,3.0) (21,3.0) (22,3.0) (23,3.0) (24,5.0) (25,5.0) (26,5.0) (27,5.0) (28,5.0) (29,7.0) (30,7.0) (31,7.0) (32,7.0) (33,7.0) (34,7.0) (35,7.0) (36,7.0) (37,9.0) (38,9.0) (39,9.0) (40,9.0) (41,9.0) (42,9.0) (43,11.0) (44,11.0) (45,11.0) (46,11.0) (47,11.0) (48,11.0) (49,11.0) (50,11.0) (51,11.0) (52,13.0) (53,13.0) (54,13.0) (55,13.0) (56,13.0) (57,13.0) (58,13.0) (59,13.0) (60,13.0) (61,13.0) (62,13.0) (63,15.0) (64,15.0) (65,15.0) (66,15.0) (67,15.0) (68,15.0) (69,15.0) (70,15.0) (71,15.0) (72,15.0) (73,15.0) (74,15.0) (75,15.0) (76,15.0) (77,15.0) (78,15.0) (79,15.0) (80,15.0) (81,15.0) (82,15.0) (83,15.0) (84,15.0) (85,15.0) (86,15.0) (87,15.0) (88,15.0) (89,15.0) (90,15.0) (91,15.0) (92,31.0) (93,33.0) (94,41.0) (95,43.0) (96,43.0) (97,43.0) (98,45.0) (99,45.0) (100,45.0) (101,45.0) (102,45.0) (103,62.0) (104,62.0) (105,64.0) (106,68.0) (107,70.0) (108,72.0) (109,90.0) (110,90.0) (111,90.0) (112,90.0) (113,90.0) (114,90.0) (115,90.0) (116,141.0) (117,141.0) (118,172.0) (119,180.0) (120,180.0) (121,180.0) (122,180.0) (123,180.0) (124,180.0) (125,180.0) (126,180.0) (127,180.0) (128,180.0) (129,254.0) (130,276.0) (131,284.0) (132,284.0) (133,291.0) (134,299.0) (135,306.0) (136,352.0) (137,352.0) (138,360.0) (139,360.0) (140,360.0) (141,360.0) (142,360.0) (143,494.0) (144,494.0) (145,509.0) (146,516.0) (147,539.0) (148,539.0) (149,546.0) (150,561.0) (151,584.0) (152,599.0) (153,606.0) (154,629.0) (155,636.0) (156,666.0) (157,666.0) (158,681.0) (159,681.0) (160,689.0) (161,689.0) (162,720.0) (163,720.0) (164,720.0) (165,720.0) (166,720.0) (167,720.0) (168,720.0) (169,720.0) (170,720.0) (171,720.0) (172,720.0) (173,720.0) (174,720.0) (175,720.0) (176,966.0) (177,974.0) (178,974.0) (179,974.0) (180,1026.0) (181,1049.0) (182,1064.0) (183,1079.0) (184,1116.0) (185,1116.0) (186,1221.0) (187,1341.0) (188,1386.0) (189,1409.0) (190,1416.0) (191,1424.0) (192,1424.0) (193,1432.0) (194,1440.0) (195,1440.0) (196,1440.0) (197,1440.0) (198,1440.0) (199,1440.0) (200,1440.0) (201,1440.0) (202,1440.0) (203,1934.0) (204,1979.0) (205,2136.0) (206,2256.0) (207,2451.0) (208,2474.0) (209,2519.0) (210,2556.0) (211,2774.0) (212,2774.0) (213,2880.0) (214,2880.0) (215,2880.0) (216,2880.0) (217,2880.0) (218,2880.0) (219,2880.0) (220,2880.0) (221,2880.0) (222,3936.0) (223,3974.0) (224,4131.0) (225,4191.0) (226,4274.0) (227,4311.0) (228,4349.0) (229,4416.0) (230,4664.0) (231,5024.0) (232,5024.0) (233,5069.0) (234,5264.0) (235,5399.0) (236,5444.0) (237,5534.0) (238,5661.0) (239,5760.0) (240,5760.0) (241,5760.0) (242,5760.0) (243,5760.0) (244,5760.0) (245,5760.0) (246,5760.0) (247,5760.0) (248,7761.0) (249,7904.0) (250,7964.0) (251,8114.0) (252,8699.0) (253,8759.0) (254,8871.0) (255,11520.0) (256,11520.0) (257,11520.0) (258,15854.0) (259,16341.0) (260,16349.0) (261,16446.0) (262,16506.0) (263,16544.0) (264,16589.0) (265,16619.0) (266,16649.0) (267,16724.0) (268,16829.0) (269,16844.0) (270,21959.0) (271,22071.0) (272,22319.0) (273,22446.0) (274,22889.0) (275,22889.0) (276,22896.0) (277,23040.0) (278,23040.0) (279,23040.0) (280,nan) (281,nan) (282,nan) (283,nan) (284,nan) (285,nan) (286,nan) (287,nan) (288,nan) (289,nan) (290,nan) (291,nan) (292,nan) (293,nan) (294,nan) (295,nan) (296,nan) (297,nan) (298,nan) (299,nan) (300,nan) (301,nan) (302,nan) (303,nan) (304,nan) (305,nan) (306,nan) (307,nan) (308,nan) (309,nan) (310,nan) (311,nan) (312,nan) (313,nan) (314,nan) (315,nan) (316,nan) (317,nan) (318,nan) (319,nan) (320,nan) (321,nan) (322,nan) (323,nan) (324,nan) (325,nan) (326,nan) (327,nan) (328,nan) (329,nan) (330,nan) (331,nan) (332,nan) (333,nan) (334,nan) (335,nan) (336,nan) (337,nan) (338,nan) (339,nan) (340,nan) (341,nan) (342,nan)};%
	\addlegendentry{examples/}%
	\addplot coordinates {(0,1.0) (1,1.0) (2,1.0) (3,1.0) (4,1.0) (5,1.0) (6,1.0) (7,1.0) (8,1.0) (9,1.0) (10,1.0) (11,1.0) (12,1.0) (13,1.0) (14,1.0) (15,1.0) (16,1.0) (17,1.0) (18,1.0) (19,1.0) (20,1.0) (21,1.0) (22,1.0) (23,1.0) (24,1.0) (25,1.0) (26,1.0) (27,1.0) (28,1.0) (29,1.0) (30,1.0) (31,1.0) (32,1.0) (33,1.0) (34,1.0) (35,1.0) (36,1.0) (37,1.0) (38,1.0) (39,1.0) (40,1.0) (41,1.0) (42,1.0) (43,1.0) (44,1.0) (45,1.0) (46,1.0) (47,1.0) (48,1.0) (49,1.0) (50,3.0) (51,5.0) (52,5.0) (53,5.0) (54,5.0) (55,9.0) (56,11.0) (57,11.0) (58,13.0) (59,15.0) (60,15.0) (61,15.0) (62,15.0) (63,15.0) (64,15.0) (65,15.0) (66,15.0) (67,15.0) (68,15.0) (69,15.0) (70,15.0) (71,15.0) (72,15.0) (73,15.0) (74,15.0) (75,15.0) (76,15.0) (77,15.0) (78,15.0) (79,15.0) (80,15.0) (81,15.0) (82,15.0) (83,15.0) (84,15.0) (85,15.0) (86,15.0) (87,15.0) (88,15.0) (89,15.0) (90,15.0) (91,33.0) (92,33.0) (93,33.0) (94,33.0) (95,35.0) (96,37.0) (97,37.0) (98,37.0) (99,37.0) (100,43.0) (101,43.0) (102,45.0) (103,45.0) (104,45.0) (105,45.0) (106,88.0) (107,88.0) (108,88.0) (109,88.0) (110,88.0) (111,90.0) (112,180.0) (113,180.0) (114,180.0) (115,284.0) (116,284.0) (117,284.0) (118,284.0) (119,284.0) (120,284.0) (121,291.0) (122,291.0) (123,291.0) (124,291.0) (125,291.0) (126,314.0) (127,352.0) (128,352.0) (129,352.0) (130,352.0) (131,352.0) (132,352.0) (133,352.0) (134,360.0) (135,360.0) (136,360.0) (137,360.0) (138,360.0) (139,501.0) (140,501.0) (141,501.0) (142,501.0) (143,501.0) (144,501.0) (145,509.0) (146,509.0) (147,509.0) (148,509.0) (149,509.0) (150,539.0) (151,539.0) (152,539.0) (153,539.0) (154,539.0) (155,539.0) (156,539.0) (157,539.0) (158,554.0) (159,554.0) (160,554.0) (161,561.0) (162,576.0) (163,591.0) (164,591.0) (165,621.0) (166,666.0) (167,666.0) (168,681.0) (169,696.0) (170,696.0) (171,696.0) (172,696.0) (173,712.0) (174,720.0) (175,720.0) (176,720.0) (177,720.0) (178,720.0) (179,720.0) (180,720.0) (181,720.0) (182,720.0) (183,720.0) (184,720.0) (185,720.0) (186,720.0) (187,720.0) (188,720.0) (189,720.0) (190,720.0) (191,720.0) (192,720.0) (193,720.0) (194,720.0) (195,720.0) (196,720.0) (197,1004.0) (198,1004.0) (199,1004.0) (200,1064.0) (201,1094.0) (202,1124.0) (203,1124.0) (204,1124.0) (205,1124.0) (206,1124.0) (207,1169.0) (208,1169.0) (209,1169.0) (210,1169.0) (211,1206.0) (212,1229.0) (213,1229.0) (214,1229.0) (215,1229.0) (216,1281.0) (217,1281.0) (218,1440.0) (219,1440.0) (220,1440.0) (221,1440.0) (222,1440.0) (223,1440.0) (224,1440.0) (225,1440.0) (226,1440.0) (227,1440.0) (228,1926.0) (229,1979.0) (230,1979.0) (231,2286.0) (232,2286.0) (233,2331.0) (234,2331.0) (235,2369.0) (236,2511.0) (237,2556.0) (238,2880.0) (239,2880.0) (240,2880.0) (241,2880.0) (242,2880.0) (243,2880.0) (244,2880.0) (245,2880.0) (246,2880.0) (247,2880.0) (248,2880.0) (249,4064.0) (250,4064.0) (251,4064.0) (252,4064.0) (253,4311.0) (254,4979.0) (255,4979.0) (256,4979.0) (257,5069.0) (258,5369.0) (259,5369.0) (260,5369.0) (261,5369.0) (262,5444.0) (263,5579.0) (264,5579.0) (265,5616.0) (266,5616.0) (267,5616.0) (268,5616.0) (269,5760.0) (270,5760.0) (271,5760.0) (272,5760.0) (273,5760.0) (274,5760.0) (275,5760.0) (276,5760.0) (277,5760.0) (278,5760.0) (279,5760.0) (280,7799.0) (281,7799.0) (282,7799.0) (283,7964.0) (284,8174.0) (285,8174.0) (286,8174.0) (287,8699.0) (288,8729.0) (289,8729.0) (290,9066.0) (291,9066.0) (292,9066.0) (293,9291.0) (294,9291.0) (295,11520.0) (296,11520.0) (297,11520.0) (298,16386.0) (299,16401.0) (300,16476.0) (301,16499.0) (302,16506.0) (303,16581.0) (304,16604.0) (305,16634.0) (306,16739.0) (307,16769.0) (308,21899.0) (309,22146.0) (310,22244.0) (311,22259.0) (312,22364.0) (313,22364.0) (314,22439.0) (315,23040.0) (316,23040.0) (317,23040.0) (318,23040.0) (319,23040.0) (320,nan) (321,nan) (322,nan) (323,nan) (324,nan) (325,nan) (326,nan) (327,nan) (328,nan) (329,nan) (330,nan) (331,nan) (332,nan) (333,nan) (334,nan) (335,nan) (336,nan) (337,nan) (338,nan) (339,nan) (340,nan) (341,nan) (342,nan)};%
	\addlegendentry{epmc/}%
	\addplot coordinates {(0,1.0) (1,1.0) (2,1.0) (3,1.0) (4,1.0) (5,1.0) (6,1.0) (7,1.0) (8,1.0) (9,1.0) (10,3.0) (11,3.0) (12,3.0) (13,3.0) (14,3.0) (15,3.0) (16,5.0) (17,5.0) (18,5.0) (19,5.0) (20,5.0) (21,5.0) (22,5.0) (23,5.0) (24,5.0) (25,5.0) (26,5.0) (27,5.0) (28,5.0) (29,5.0) (30,5.0) (31,5.0) (32,5.0) (33,5.0) (34,5.0) (35,5.0) (36,7.0) (37,7.0) (38,7.0) (39,7.0) (40,7.0) (41,7.0) (42,7.0) (43,7.0) (44,7.0) (45,7.0) (46,7.0) (47,7.0) (48,7.0) (49,7.0) (50,7.0) (51,7.0) (52,7.0) (53,7.0) (54,7.0) (55,7.0) (56,7.0) (57,7.0) (58,7.0) (59,7.0) (60,7.0) (61,9.0) (62,9.0) (63,9.0) (64,9.0) (65,9.0) (66,9.0) (67,9.0) (68,9.0) (69,9.0) (70,9.0) (71,9.0) (72,9.0) (73,9.0) (74,9.0) (75,9.0) (76,11.0) (77,11.0) (78,11.0) (79,11.0) (80,11.0) (81,11.0) (82,11.0) (83,11.0) (84,13.0) (85,15.0) (86,15.0) (87,15.0) (88,15.0) (89,15.0) (90,43.0) (91,43.0) (92,43.0) (93,43.0) (94,43.0) (95,43.0) (96,45.0) (97,45.0) (98,45.0) (99,45.0) (100,45.0) (101,45.0) (102,78.0) (103,78.0) (104,78.0) (105,78.0) (106,78.0) (107,78.0) (108,84.0) (109,84.0) (110,84.0) (111,84.0) (112,84.0) (113,90.0) (114,180.0) (115,180.0) (116,180.0) (117,180.0) (118,180.0) (119,180.0) (120,180.0) (121,180.0) (122,180.0) (123,180.0) (124,180.0) (125,180.0) (126,180.0) (127,180.0) (128,180.0) (129,180.0) (130,180.0) (131,180.0) (132,180.0) (133,180.0) (134,180.0) (135,180.0) (136,314.0) (137,329.0) (138,344.0) (139,344.0) (140,344.0) (141,344.0) (142,344.0) (143,352.0) (144,352.0) (145,360.0) (146,360.0) (147,360.0) (148,360.0) (149,360.0) (150,360.0) (151,360.0) (152,360.0) (153,360.0) (154,360.0) (155,360.0) (156,360.0) (157,360.0) (158,360.0) (159,360.0) (160,360.0) (161,524.0) (162,546.0) (163,569.0) (164,569.0) (165,569.0) (166,591.0) (167,606.0) (168,621.0) (169,651.0) (170,651.0) (171,651.0) (172,651.0) (173,659.0) (174,659.0) (175,674.0) (176,689.0) (177,712.0) (178,712.0) (179,720.0) (180,720.0) (181,720.0) (182,720.0) (183,720.0) (184,720.0) (185,720.0) (186,720.0) (187,720.0) (188,966.0) (189,989.0) (190,1019.0) (191,1086.0) (192,1094.0) (193,1199.0) (194,1304.0) (195,1386.0) (196,1424.0) (197,1440.0) (198,1440.0) (199,1440.0) (200,1440.0) (201,1440.0) (202,1440.0) (203,1440.0) (204,1440.0) (205,1440.0) (206,1440.0) (207,1440.0) (208,1440.0) (209,1440.0) (210,1440.0) (211,2016.0) (212,2159.0) (213,2159.0) (214,2159.0) (215,2159.0) (216,2159.0) (217,2159.0) (218,2166.0) (219,2211.0) (220,2211.0) (221,2211.0) (222,2369.0) (223,2376.0) (224,2384.0) (225,2534.0) (226,2706.0) (227,2706.0) (228,2706.0) (229,2774.0) (230,2774.0) (231,2774.0) (232,2819.0) (233,2849.0) (234,2864.0) (235,2864.0) (236,2872.0) (237,2880.0) (238,2880.0) (239,2880.0) (240,2880.0) (241,2880.0) (242,2880.0) (243,2880.0) (244,3846.0) (245,4124.0) (246,4124.0) (247,4124.0) (248,4124.0) (249,4281.0) (250,4334.0) (251,4476.0) (252,4799.0) (253,4859.0) (254,4904.0) (255,4904.0) (256,4926.0) (257,4979.0) (258,5031.0) (259,5031.0) (260,5031.0) (261,5031.0) (262,5069.0) (263,5144.0) (264,5159.0) (265,5219.0) (266,5609.0) (267,5616.0) (268,5624.0) (269,5624.0) (270,5624.0) (271,5624.0) (272,5760.0) (273,5760.0) (274,5760.0) (275,5760.0) (276,5760.0) (277,7694.0) (278,7829.0) (279,7829.0) (280,7836.0) (281,8084.0) (282,8646.0) (283,8886.0) (284,9456.0) (285,9599.0) (286,10319.0) (287,10566.0) (288,10604.0) (289,10694.0) (290,10859.0) (291,10911.0) (292,11031.0) (293,11520.0) (294,11520.0) (295,16221.0) (296,16386.0) (297,16401.0) (298,16476.0) (299,16499.0) (300,16506.0) (301,16581.0) (302,16604.0) (303,16634.0) (304,16739.0) (305,16769.0) (306,17421.0) (307,18741.0) (308,19124.0) (309,21899.0) (310,22146.0) (311,22244.0) (312,22259.0) (313,22439.0) (314,23040.0) (315,23040.0) (316,23040.0) (317,23040.0) (318,23040.0) (319,nan) (320,nan) (321,nan) (322,nan) (323,nan) (324,nan) (325,nan) (326,nan) (327,nan) (328,nan) (329,nan) (330,nan) (331,nan) (332,nan) (333,nan) (334,nan) (335,nan) (336,nan) (337,nan) (338,nan) (339,nan) (340,nan) (341,nan) (342,nan)};%
	\addlegendentry{ltl2ldba/}%
\end{axis}
\end{tikzpicture}




\end{frame}
\begin{frame}{Porovnání vlivu na zpětnovazebné učení}
	Neúspěch nastane v případě, že alespoň jednou pro příklad není dosaženo pravděpodobnosti 1
	\begin{table}[ht]
		\centering
		
		\caption{Počty nejlepších výsledků z Mungojerrie }
		\label{Table:mungostatsnoex}
		\input{stats-noexamples.tex}
	\end{table}
\end{frame}

\begin{frame}[plain]
	\vfill
	\centerline{Děkuji za Vaši pozornost}
	\vfill\vfill
\end{frame}





	\begin{frame}{Reakce na posudek vedoucího}
		\begin{itemize}
		\item $\ldots$Navíc se domnívám, že zmíněný příklad \emph{Figure 4.1} je špatně, nebot’ barevně naznačuje existenci slabých slim hran, které však v tomto automatu nemohou být, protože ve všech stavech platí $S'= \emptyset$
		\begin{itemize}
		\item Figure 4.1 -> Barvy přechodů také zvýrazňují úrovně, barvy jako akceptační podmínky ze zdrojového automatu. Není zde spojitost s obarvením hran z přechozího příkladu. Automat není špatně, nemá žádné slim hrany, tudíž je stejný pro weak i strong slim automaty.
		\end{itemize}
	\end{itemize}

		
		
	\end{frame}
	
	\begin{frame}{Reakce na posudek vedoucího}		
		Otázky:
		\begin{enumerate}
			\item Tabulka 6.2 uvádí, že Seminator při volbě via tba na formulích z literatury produkuje silné slim automaty s 431 stavy, zatímco při volbě best produkuje automaty s 436 stavy. Jak je to možné, když volba best má vždy produkovat automat s nejvýše tolika stavy jako volba via tba?
			\begin{itemize}
				\item To je překlep, nastal při přepisu přepočítaných hodnot, má to být v obou případech 436.
			\end{itemize}
			\pause
			\item Proč je v tabulce 6.7 u nástroje Seminator slim best uveden nižší počet stav u než v tabulce 6.2 pro volby best/best? Čekal bych identické hodnoty.
			\begin{itemize}
				\item Srovnáváme pouze automaty kde všechny nástroje dokončí výpočet.
			\end{itemize}
			
			\pause
			\item Proč se liší hodnoty na řádcích failures v tabulkách 7.1 a 7.2?
			\begin{itemize}
				\item V sedmé kapitole mám zmíněno "We exclude uninteresting benchmarks, where all tools achieve the same result." 
				Odebráním jednoho nástroje vznikly výsledky, kde mají všechny zbývající nástroje stejný výsledek.
			\end{itemize}
			
		\end{enumerate}
		
		
	\end{frame}
	
	\begin{frame}{Reakce na posudek oponenta}{Změny v kódu}
		'V přiloženém ZIP archivu jsem mezi všemi soubory našel seminator-ba2slim/src/slim.hpp o 39 řádcích (včetně prázdných),  který má v záhlaví uvedeno  „Created by psimovec on 9/16/20“.  Ostatní soubory vypadají, že jsou v nástroji Seminátor původní.'
		
		Změny v kódu:
		
	\begin{table}[ht]
		\centering
		
		
		
		\begin{tabular}{|c|c|}
			\hline
			soubor  &   změny \\
			\hline
			Makefile.am & +1\\
			\hlineny
			tests/slim.test          &       +16 \\
			
			\hline
			src/breakpoint\_twa.cpp    &     +18 \\
			\hline
			src/breakpoint\_twa.hpp      &        +16 \\
			\hlineny
			src/main.cpp&  	+95\\\hlineny
			src/slim.hpp&+39\\\hline
			
		\end{tabular}
	\end{table}
		Z pull requestu: https://github.com/mklokocka/seminator/pull/31/files
		
					
		Dále jsem vytvořil skripty na evaluaci.
	\end{frame}

	\begin{frame}{Reakce na posudek oponenta}{Literature/random datasety}
		'V tabulce 6.1 netuším, co je „literature“ a co je „random“.'
		
		Tyto datasety byly již v seminator-evaluation, jedná se o nesemideterministické automaty
		
		
		{https://github.com/xblahoud/seminator-evaluation}
		
		{https://github.com/xblahoud/seminator-evaluation/blob/master/Formulae.ipynb}
		
	\end{frame}
	
	\section{\bibname}
	\begin{frame}[t, allowframebreaks]{\bibname}
		\printbibliography[heading=none]
	\end{frame}
	
	
\end{document}